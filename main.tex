\documentclass{beamer}

\usepackage[style=numeric,maxnames=2]{biblatex}
\usepackage{minted}
\usepackage{tikz}

\definecolor{lightgray}{gray}{0.9}
\definecolor{mediumslateblue}{HTML}{7B68EE}
\newcommand*\circled[1]{\tikz[baseline=(char.base)]{\node[shape=circle,fill,inner sep=2pt] (char) {\textcolor{white}{#1}};}} % chktex 36

\usepackage[orientation=portrait,size=a1,scale=1.0]{beamerposter}
\usetheme{JuelichPoster}

\usetikzlibrary{positioning}

\ExecuteBibliographyOptions{%
  sorting=nyt,
  bibwarn=true,
  isbn=false,
  url=true%
}
\addbibresource{references.bib}

\setbeamertemplate{partner1}{\includegraphics{img/cscs}}
\setbeamertemplate{partner2}{\includegraphics{img/HBP_logo}}

\begin{document}
\begin{frame}[t, fragile]
  \frametitle{\includegraphics[width=0.66\linewidth]{img/arbor-lines-proto-colour-full}}
  \framesubtitle{Generating High Performance Simulations from a Portable Data Format using Arbor\\
    \tiny{T. Hater, B. Huisman, L. Landsmeer (Forschungszentrum Jülich)}}
  \begin{columns}[onlytextwidth,T]
    \begin{column}{0.65\textwidth}
      Computational neuroscience is experiencing a steady growth in available
      simulation tools applicable to morphologically detailed cell descriptions
      \cite{eden, neurogpu, carl, brian, arb, nrn}. However, the development of
      models that are actually portable between simulators lags behind. NeuroML2
      is one of the few comprehensive approaches in this area, but its reference
      implementation lacks in performance and scalability to larger simulations
      \cite{nml2}. Our goal is to enable Arbor, a performance-portable
      library for simulating morphologically detailed neurons to consume NeuronML2
      models. We present \texttt{nmlcc}, a tool to generate optimised,
      full scale simulations from a description in NeuroML2 \cite{nmlcc}. It
      produces bespoke dynamics tailored to the input, resulting in performance
      metrics comparable to hand-optimized code. Through Arbor, the generated
      simulation package is able to utilize modern hardware, including
      large-scale GPU clusters, scaling to millions of cells \cite{arb}. As a
      case study, we show how a single cell simulation based on \cite{Hay} was
      ported to Arbor using \texttt{nmlcc}. Furthermore, we analyse the runtime
      performance of the produced model.
    \end{column}
    \begin{column}{0.3\textwidth}
      \vspace*{-1ex}
      \begin{block}{Where to find us}
        \begin{description}
          \item[Arbor] \href{https://arbor-sim.github.io}{arbor-sim.github.io}
          \item[nmlcc] \href{https://github.com/thorstenhater/nmlcc}{github.com/thorstenhater/nmlcc}
          \item[Contact] \href{mailto:t.hater@fz-juelich.de}{t.hater@fz-juelich.de}
        \end{description}
      \end{block}
    \end{column}
  \end{columns}
  % problem statement
  \vspace*{2ex}
  \begin{columns}
    \begin{column}{0.49\textwidth}
      \textbf{Dynamic Compilation}

      NeuroML2 (NML) constitutes a data language for describing a full neuronal
      network, including network connectivity, cell morphologies,
      parametrisation, and ion channel dynamics. NML is defined as a library of
      LEMS specifications, which can be dynamically extended while building a
      simulation.

      This constrains our choice of implementation, since we cannot deliver a
      pre-built set of ion channels as an addition to Arbor, but must be able to
      synthesise these on demand. On the other hand, having the full simulation
      available for introspection enables optimisations that go beyond Arbor's
      builtin capabilities.
    \end{column}
    \begin{column}{0.49\textwidth}

    \end{column}
  \end{columns}
  % Data flow
  \vspace*{2ex}
  \begin{columns}
    \begin{column}{0.49\textwidth}
      \begin{tikzpicture}
        % input
        \node[draw, rectangle, rounded corners, ultra thick, color=black!80, text width=8cm] (nml) {\verb!<neuroml>!\\\verb!  ...!\\\verb!</neuroml>!};
        % nmlcc products
        \node[draw, rectangle, rounded corners, ultra thick, color=black!80, text width=5cm, below left=3cm and -3.5cm of nml] (mrf) {\verb!mrf/morph-0.nml!\\\verb!mrf/morph-1.nml!};
        \node[draw, rectangle, rounded corners, ultra thick, color=black!80, text width=4.5cm, below right=3cm and -3.5cm of nml] (acc) {\verb!acc/cell-0.acc!\\\verb!acc/cell-1.acc!};
        \node[draw, rectangle, rounded corners, ultra thick, color=black!80, text width=5cm, right=1cm of acc]   (sim) {\verb!main.py!\\\verb!import arbor as A!};
        \node[draw, rectangle, rounded corners, ultra thick, color=black!80, text width=4cm, left=1cm of mrf] (cat) {\verb!cat/hh.mod!\\\verb!cat/leak.mod!};
        % connections from nml to outputs
        \node[below=2cm of nml] (h0) {};
        \draw[ultra thick, minimum size=0pt] (nml.south) -- (h0.center) node[right, pos=0.5]{\verb!nmlcc bundle!};
        \draw[->, >=stealth, ultra thick, minimum size=0pt] (h0.center) -| (acc.north);
        \draw[->, >=stealth, ultra thick, minimum size=0pt] (h0.center) -| (sim.north);
        \draw[->, >=stealth, ultra thick, minimum size=0pt] (h0.center) -| (mrf.north);
        \draw[->, >=stealth, ultra thick, minimum size=0pt] (h0.center) -| (cat.north);
        % connection from outputs to sim
        \node[below=1cm of sim] (h1) {};
        \draw[ultra thick, ->, >=stealth] (acc.south) |- ([xshift=-10pt, yshift=10pt]h1.center)  -| ([xshift=-10pt]sim.south);
        \draw[ultra thick, ->, >=stealth] (mrf.south) |- (h1.center) -| (sim.south);
        \draw[ultra thick, ->, >=stealth] (cat.south) |- ([xshift=10pt, yshift=-10pt]h1.center) node[below, pos=0.75] {read} -| ([xshift=10pt]sim.south);
      \end{tikzpicture}
    \end{column}
    \begin{column}{0.49\textwidth}
      \textbf{Dataflow}

      \texttt{nmlcc} consumes NML2 descriptions at the level of individual ion
      channels, cells -- comprising a morphology, ion channles, and parameter
      assignments --, and full networks, which include cells, their
      connectivity, and network-level parameters. The latter subsumes all
      previous modes and generates an executable Python script to instantiate a
      simulation in Arbor. This is achieved by turning the components into files
      in formats Arbor understands and connecting them in Python.
      \begin{description}
        \item[Ion channel] Transformed to NMODL;\@ all channels will be bundled and compiled to a
        shared library that Arbor can hook into.
        \item[Morphology] Extracted to standalone NML2 morphology files, which Arbor can
        load natively.
        \item[Assignments] ACC, Arbor's cable cell serialisation format.
              Includes passive parameters and ion channel mappings.
        \item[Connectivity] Stored inside the Python script as dictionaries.
        \item[Stimuli] Added by the Python script.
      \end{description}

      \texttt{nmlcc}'s internal data model is not hand-written, instead it will
      be automatically generated from the schemea files and base libraries
      included with NML2. This two-phase approach allows us to rapidly adapt to
      changes in NML2.
    \end{column}
  \end{columns}
  % Data flow
  \vspace*{2ex}
  \begin{columns}
    \begin{column}{0.49\textwidth}
      \textbf{Optimisations}

      Arbor's model for building simulations is highly flexible: cell types and
      their ion channel dynamics are given at runtime. Ion channels can be
      loaded from shared libraries at start-up. However, this flexibility comes
      at the cost of missing performance optimisations.

      Since \texttt{nmlcc} can inspect the full simulation, it has extensive
      potential for optimisation, mainly when dealing with ion channel
      descriptions. These are one of the two main cost centres in a simulation,
      the other is solving the cable equation. We perform two central high-level
      transformations: Specialisation and Combination. Both are designed to
      expose more optimisations to the NMODL-to-C++ and C++-to-binary compiler.

      We exploit that NML2 provides assignments of the form
      \verb!(region, channel, parameters)! where
      \begin{description}
        \item[\texttt{region}] sub-set of the morphology's segments
        \item[\texttt{channel}] the ion channel name
        \item[\texttt{parameters}] a list of key-value pairs to be set on the channel
      \end{description}
    \end{column}
    \begin{column}{0.49\textwidth}
      \emph{Specialisation}

      Since NML2 defines ion channels as parametrised templates, we can generate
      one copy of each assigned channel per region where the parameters have
      been fixed to their concrete values for this region. This allows for
      replacing memory accesses by constants and folding multiple constants into
      one. Then, all operations on these constant values are evaluated and
      conditionals are replaced by the appropriate branch where possible.

      \vspace*{1ex}
      \emph{Combination or Super-Mechanisms}

      Using the same information, we can also combine all ion channels added to
      the same region in to a single one, dubbed a `super-mechanism'. As Arbor's
      ion channels are implemented as a set of callbacks on a standardised set
      of parameters, this reduces the amount of functions called and data moved.

      \vspace*{1ex}
      \emph{General Optimisation of Ion Channels}

      When generating NMODL from NML2 we also apply a series of guidelines we
      found helpful to achive good performance. These mainly minimise the amount
      of data movement;\@ an overview can be found in Arbor's documentation.
    \end{column}
  \end{columns}
\end{frame}
\end{document}
