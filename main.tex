\documentclass{beamer}

\usepackage[style=numeric,maxnames=2]{biblatex}
\usepackage{minted}
\usepackage{tikz}

\definecolor{lightgray}{gray}{0.9}
\definecolor{mediumslateblue}{HTML}{7B68EE}
\newcommand*\circled[1]{\tikz[baseline=(char.base)]{\node[shape=circle,fill,inner sep=2pt] (char) {\textcolor{white}{#1}};}} % chktex 36

\usepackage[orientation=portrait,size=a1,scale=1.0]{beamerposter}
\usetheme{JuelichPoster}

\ExecuteBibliographyOptions{%
  sorting=nyt,
  bibwarn=true,
  isbn=false,
  url=true%
}
\addbibresource{references.bib}

\setbeamertemplate{partner1}{\includegraphics{img/cscs}}
\setbeamertemplate{partner2}{\includegraphics{img/HBP_logo}}

\begin{document}
\begin{frame}[t, fragile]
  \frametitle{\includegraphics[width=0.66\linewidth]{img/arbor-lines-proto-colour-full}}
  \framesubtitle{Generating High Performance Simulations from a Portable Data Format using Arbor\\
    \tiny{T. Hater, B. Huisman, L. Landsmeer (Forschungszentrum Jülich)}}
  \begin{columns}[onlytextwidth,T]
    \begin{column}{0.65\textwidth}
      Computational neuroscience is experiencing a steady growth in available
      simulation tools applicable to morphologically detailed cell descriptions
      \cite{eden, neurogpu, carl, brian, arb, nrn}. However, the development of
      models that are actually portable between simulators lags behind. NeuroML2
      is one of the few comprehensive approaches in this area, but its reference
      implementation lacks in performance and scalability to larger simulations
      \cite{nml2}. Our goal is to enable Arbor, a performance-portable
      library for simulating morphologically detailed neurons to consume NeuronML2
      models. We present \texttt{nmlcc}, a tool to generate optimised,
      full scale simulations from a description in NeuroML2 \cite{nmlcc}. It
      produces bespoke dynamics tailored to the input, resulting in performance
      metrics comparable to hand-optimized code. Through Arbor, the generated
      simulation package is able to utilize modern hardware, including
      large-scale GPU clusters, scaling to millions of cells \cite{arb}. As a
      case study, we show how a single cell simulation based on \cite{Hay} was
      ported to Arbor using \texttt{nmlcc}. Furthermore, we analyse the runtime
      performance of the produced model.
    \end{column}
    \begin{column}{0.3\textwidth}
      \vspace*{-1ex}
      \begin{block}{Where to find us}
        \begin{description}
          \item[Arbor] \href{https://arbor-sim.github.io}{arbor-sim.github.io}
          \item[nmlcc] \href{https://github.com/thorstenhater/nmlcc}{github.com/thorstenhater/nmlcc}
          \item[Contact] \href{mailto:t.hater@fz-juelich.de}{t.hater@fz-juelich.de}
        \end{description}
      \end{block}
    \end{column}
  \end{columns}
  % problem statement
  \vspace*{2ex}
  \begin{columns}
    \begin{column}{0.49\textwidth}
      \textbf{Dynamic Compilation}

      NeuroML2 (NML) constitutes a data language for describing a full neuronal
      network, including network connectivity, cell morphologies,
      parametrisation, and ion channel dynamics. NML is defined as a library of
      LEMS specifications, which can be dynamically extended while building a
      simulation.

      This constrains our choice of implementation, since we cannot deliver a
      pre-built set of ion channels as an addition to Arbor, but must be able to
      synthesise these on demand. On the other hand, having the full simulation
      available for introspection enables optimisations that go beyond Arbor's
      builtin capabilities.
    \end{column}
    \begin{column}{0.49\textwidth}

    \end{column}
  \end{columns}
  % Data flow
  \vspace*{2ex}
  \begin{columns}
    \begin{column}{0.49\textwidth}
      \textbf{\texttt{nmlcc} at Work}

      \texttt{nmlcc}
    \end{column}
    \begin{column}{0.49\textwidth}

    \end{column}
  \end{columns}

\end{frame}
\end{document}
