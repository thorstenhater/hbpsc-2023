\documentclass{beamer}

\usepackage[style=numeric,maxnames=2]{biblatex}
\usepackage{minted}
\usepackage{tikz}

\definecolor{lightgray}{gray}{0.9}
\definecolor{mediumslateblue}{HTML}{7B68EE}
\newcommand*\circled[1]{\tikz[baseline=(char.base)]{\node[shape=circle,fill,inner sep=2pt] (char) {\textcolor{white}{#1}};}} % chktex 36

\usepackage[orientation=portrait,size=a1,scale=1.0]{beamerposter}
\usetheme{JuelichPoster}

\ExecuteBibliographyOptions{%
  sorting=nyt,
  bibwarn=true,
  isbn=false,
  url=true%
}
\addbibresource{references.bib}

\setbeamertemplate{partner1}{\includegraphics{img/cscs}}
\setbeamertemplate{partner2}{\includegraphics{img/HBP_logo}}

\begin{document}
\begin{frame}[t, fragile]
  \frametitle{\includegraphics[width=0.66\linewidth]{img/arbor-lines-proto-colour-full}}
  \framesubtitle{Generating High Performance Simulations from a Portable Data Format using Arbor\\
    \tiny{T. Hater, B. Huisman, L. Landsmeer (Forschungszentrum Jülich)}}
  \begin{columns}[onlytextwidth,T]
    \begin{column}{0.65\textwidth}
      Computational neuroscience is experiencing a steady growth in available
      simulation tools applicable to morphologically detailed cell descriptions
      \cite{eden, neurogpu, carl, brian, arb, nrn}. However, the development of
      models that are actually portable between simulators lags behind. NeuroML2
      is one of the few comprehensive approaches to portably describing whole
      simulations, but its reference implementation lacks in performance and
      scalability to larger simulations \cite{nml2}. Our goal is to enable
      Arbor, a modern, performance-portable library for simulating
      morphologically detailed neurons to consume models in NeuronML2. We
      present \texttt{nmlcc}, a tool to generate optimised, full scale
      simulations from a description in NeuroML2 \cite{nmlcc}. It produces
      bespoke dynamics tailored to the input, resulting in performance metrics
      comparable to hand-optimized code. Through Arbor, the generated simulation
      package is able to utilize modern hardware, including large-scale GPU
      clusters, scaling to millions of cells \cite{arb}. As a case study, we
      show how a single cell simulation based on \cite{Hay} was ported to Arbor
      using \texttt{nmlcc}. Furthermore, we analyse the runtime performance of
      the produced model.
    \end{column}
    \begin{column}{0.3\textwidth}
      \vspace*{-1ex}
      \begin{block}{Where to find us}
        \begin{description}
          \item[Website] \href{https://arbor-sim.github.io}{arbor-sim.github.io}
          \item[Source code] \href{https://github.com/arbor-sim/arbor}{github.com/arbor-sim/arbor}
          \item[Documentation] \href{https://arbor.readthedocs.io}{arbor.readthedocs.io}
          \item[Contact] \href{mailto:arbor-sim@fz-juelich.de}{arbor-sim@fz-juelich.de}
        \end{description}
      \end{block}
    \end{column}
  \end{columns}
\end{frame}
\end{document}
